\usepackage{etex}
\usepackage{xcolor}
\usepackage[a4paper,left=3.5cm,right=2.5cm,bottom=3.5cm,top=3cm]{geometry}
\usepackage[english]{babel}
\usepackage[round]{natbib}
\usepackage{graphicx,tikz}
\usetikzlibrary{matrix,decorations.pathreplacing, calc, positioning}
\usepackage{amsmath,amssymb,nicefrac,bbm}
\usepackage{relsize} % mathlarger
\usepackage{hyperref,url}
\usepackage{silence}
\WarningFilter{latex}{Overwriting file}

\usepackage[colorinlistoftodos]{todonotes}
\reversemarginpar
\setlength{\marginparwidth}{3.1cm}
% Theorem-Umgebungen
%\usepackage[amsmath,thmmarks]{ntheorem}
\usepackage{amsthm}
\usepackage{thmtools, thm-restate}
\usepackage{algorithm,algpseudocode}
\algnewcommand{\IIf}[1]{\State\algorithmicif\ #1\ \algorithmicthen}
\algnewcommand{\EndIIf}{\unskip\ \algorithmicend\ \algorithmicif}

\usepackage{enumerate}
\usepackage{booktabs,multirow,adjustbox}
\usepackage[font=small,labelfont=bf]{caption}
\usepackage{subfigure}
\usepackage{pgfplots,filecontents,pgfplotstable}
\usepgfplotslibrary{groupplots}
\pgfplotsset{compat=1.14}
\pgfplotsset{
cohStyle/.style={ctrust,mark options ={ctrust},mark repeat={4}, ultra thick, error bars/.cd,y dir = both, y explicit},
densStyle/.style={cdens,dashed,mark options ={cdens},mark repeat={4}, ultra thick, error bars/.cd,y dir = both, y explicit},
primpStyle/.style={cPrimp,dashed,mark=triangle*,mark options ={cPrimp},mark repeat={3}, ultra thick, error bars/.cd,y dir = both, y explicit},
panStyle/.style={cPan,dashed,mark options ={cPan},mark repeat={3}, ultra thick, error bars/.cd,y dir = both, y explicit},
specStyle/.style={cSpec,mark options ={cSpec},mark repeat={4}, thick, error bars/.cd,y dir = both, y explicit},
specLStyle/.style={cSpecL,mark options ={cSpecL},mark repeat={4}, thick, error bars/.cd,y dir = both, y explicit},
RSCStyle/.style={cRSC,dashed,mark=triangle*,mark options ={cRSC},mark repeat={3}, thick, error bars/.cd,y dir = both, y explicit},
SCStyle/.style={cSC,dashed,mark options ={cSC},mark repeat={3}, thick, error bars/.cd,y dir = both, y explicit},
DBSCANStyle/.style={cDBSCAN,dashed,mark options ={cDBSCAN},mark repeat={3}, thick, error bars/.cd,y dir = both, y explicit},
clusterScatterStyle/.style={scatter/classes={1={cSpecL}, 2={cRSC}, 0={cDBSCAN}, -1={white}},
    scatter, only marks, mark size =0.3, scatter src=explicit symbolic},
nonnegScatterStyle/.style={scatter, only marks, fill=cSpec, scatter src=explicit,
     	visualization depends on ={\thisrow{z} \as \perpointmarksize},
     scatter/@pre marker code/.append style={/tikz/mark size=1.5*\perpointmarksize}},
pandaStyle/.style={cPanda,dashed,mark options ={cPanda},mark repeat={3}, ultra thick, error bars/.cd,y dir = both, y explicit},
punkStyle/.style={cPunk,mark options ={cPunk},mark repeat={3}, thick, error bars/.cd,y dir = both, y explicit},
dbssl1Style/.style={cDBSSL1,dashed,mark options ={cDBSSL1},mark=triangle,mark repeat={3}, thick, error bars/.cd,y dir = both, y explicit},
dbssl2Style/.style={cDBSSL2,dashed,mark options ={cDBSSL2},mark=diamond,mark repeat={3}, thick, error bars/.cd,y dir = both, y explicit},
mdlDbsslStyle/.style={cMDLDBSSL,dashed,mark=star,mark options ={cMDLDBSSL},mark repeat={3}, thick, error bars/.cd,y dir = both, y explicit},
% color change for embedding visualization (fuzzy clusters) 
colormap={test}{[2pt]
    rgb255=(166,206,227);
    rgb255=(166,206,227);
},
}

%List of Symbols
\usepackage{nomencl}
\makenomenclature


%\usepackage{color}
\definecolor{col1}{RGB}{166,206,227}
\definecolor{col2}{RGB}{31,120,180}
\definecolor{col3}{RGB}{178,223,138}
\definecolor{col4}{RGB}{51,160,44}
\definecolor{col5}{RGB}{251,154,153}
\definecolor{col6}{RGB}{227,26,28}

\definecolor{cdens}{named}{col1}
\definecolor{ctrust}{named}{col2}
\definecolor{cPrimp}{named}{col3}
\definecolor{cPan}{named}{col4}
\definecolor{cPanda}{named}{col5}
\definecolor{cNassau}{named}{col6}
\definecolor{cMdl4bmf}{named}{col1}


%
\definecolor{cSpec}{named}{col1}
\definecolor{cSpecL}{named}{col2}
\definecolor{cRSC}{named}{col3}
\definecolor{cSC}{named}{col4}
\definecolor{cDBSCAN}{named}{col5}

%
\definecolor{cPunk}{named}{col1}
\definecolor{cDBSSL1}{named}{col2}
\definecolor{cDBSSL2}{named}{col4}
\definecolor{cMDLDBSSL}{named}{col5}
% Theorem-Optionen %
%\theoremseparator{.}
%\newenvironment{proof}{\par\noindent{\textit{ Proof}\ }}{\hfill\qed\\[2mm]}
\theoremstyle{plain}
%\theoremheaderfont{\moon}
%\newcommand{\BlackBox}{\rule{1.5ex}{1.5ex}}  % end of proof
%
\newtheorem{theorem}{Theorem}[chapter]
\newtheorem{corollary}[theorem]{Corollary}
\newtheorem{observation}[theorem]{Observation}
\newtheorem{lemma}[theorem]{Lemma}
%\theorembodyfont{\upshape}
\theoremstyle{definition}
\newtheorem{definition}[theorem]{Definition}
\newtheorem{algSpec}{Algorithm Specification}
\newtheorem*{remark}{Remark}
\newtheorem*{example}{Example}
\newtheorem*{problem}{Problem}

%Operators/Commands
\DeclareMathOperator*{\argmin}{arg\,min}
\DeclareMathOperator*{\argmax}{arg\,max}
\DeclareMathOperator*{\freq}{freq}
\DeclareMathOperator*{\cov}{cov}
\DeclareMathOperator*{\anc}{anc}
\DeclareMathOperator*{\supp}{supp}
\DeclareMathOperator*{\minsup}{minsup}
\DeclareMathOperator*{\tr}{tr}
\DeclareMathOperator*{\bigO}{\mathcal{O}}
\DeclareMathOperator{\diag}{diag}
\DeclareMathOperator{\prox}{prox}
\DeclareMathOperator{\pre}{pre}
\DeclareMathOperator{\rec}{rec}
\newcommand{\Ya}{Y^{(a)}}
\newcommand{\Va}{V^{(a)}}
\newcommand{\Da}{D^{(a)}}
\newcommand{\KL}{Kurdyka-{\L}ojasiewicz }
\newcommand{\LXU}{\mathcal{L}}
\newcommand{\F}{\mathcal{F}}
\newcommand{\N}{\mathbb{N}}
\newcommand{\R}{\mathbb{R}}
\newcommand{\estU}{\widetilde{U}_{\widehat{CT}}}
\newcommand{\estu}{\tilde{u}_{\widehat{CT}}}
\newcommand{\estCT}{\widehat{CT}}
\newcommand{\node}{\mathfrak{n}}
\newcommand{\leaf}{\mathfrak{t}}
\newcommand{\krimp}{\textsc{Krimp} }
\newcommand{\shrimp}{\textsc{SHrimp} }
\newcommand{\slim}{\textsc{Slim} }

\makeatletter

\pgfplotstableset{
    zero color/.initial=white,
    zero color/.get=\zerocol,
    zero color/.store in=\zerocol,
    one color/.initial=red,
    one color/.get=\onecol,
    one color/.store in=\onecol,
    color cells/.style={
        every head row/.style={output empty row},
        string type,
        postproc cell content/.code={%
           \pgfkeysalso{@cell content=\rule{0cm}{2.4ex}\cellcolor{\zerocol}
           \pgfmathtruncatemacro\number{int(##1)}
           \ifnum\number>100\cellcolor{\onecol!50!black}
           \else \ifnum\number>0\cellcolor{\onecol!##1}\fi\fi}%
        },
        columns/x/.style={
            column name={},
            postproc cell content/.code={}
        }
    }
}
\makeatother

\makeatletter
\newcommand\footnoteref[1]{\protected@xdef\@thefnmark{\ref{#1}}\@footnotemark}
\makeatother

\makeatletter
%\renewcommand{\@chapapp}{}% Not necessary...
\newenvironment{chapquote}[2][2em]
  {\setlength{\@tempdima}{#1}%
   \def\chapquote@author{#2}%
   \parshape 1 \@tempdima \dimexpr\textwidth-2\@tempdima\relax%
   \itshape}
  {\par\normalfont\hfill--\ \chapquote@author\hspace*{\@tempdima}\par\bigskip}
\makeatother

%Thicker bar
\makeatletter
\newcommand{\thickbar}{\mathpalette\@thickbar}
\newcommand{\@thickbar}[2]{{#1\mkern1.5mu\vbox{
  \sbox\z@{$#1\mkern-1.5mu#2\mkern-1.5mu$}%
  \sbox\tw@{$#1\overline{#2}$}%
  \dimen@=\dimexpr\ht\tw@-\ht\z@-.8\p@\relax
  \hrule\@height.8\p@ % adjust for the desired rule thickness
  \vskip\dimen@
  \box\z@}\mkern1.5mu}
}
\makeatother

%----------box---------------
\usepackage[many]{tcolorbox}
\tcbset{
  myhlight/.style={
    colback=cyan!10,
    arc=0pt,
    outer arc=0pt,
    boxrule=0pt,
    top=2pt,
    bottom=2pt,
    left=2pt,
    right=2pt,
  },
  highlight math style={myhlight},
  mybx/.style={
    colback=white,
    arc=0pt,
    outer arc=0pt,
    boxrule=1pt,
    top=2pt,
    bottom=2pt,
    left=2pt,
    right=2pt,
  }
}

\newtcolorbox{mybox}{
	boxsep=1pt,
  	breakable,
  	mybx
}


% Zeilenabstand einstellen %
\renewcommand{\baselinestretch}{1}
% Floating-Umgebungen anpassen %
\renewcommand{\topfraction}{1.0}
\renewcommand{\bottomfraction}{1.0}
\renewcommand{\floatpagefraction}{1.0}
\renewcommand{\dblfloatpagefraction}{1.0}

% Leere Seite ohne Seitennummer, naechste Seite rechts
\newcommand{\blankpage}{
 \clearpage{\pagestyle{empty}\cleardoublepage}
}

% Keine einzelnen Zeilen beim Anfang eines Abschnitts (Schusterjungen)
\clubpenalty = 10000
% Keine einzelnen Zeilen am Ende eines Abschnitts (Hurenkinder)
\widowpenalty = 10000 \displaywidowpenalty = 10000
% EOF