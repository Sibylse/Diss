\chapter{Matrix Factorization with Binary Constraints}
\label{chap:ZeroShades}
Cholesky decomposition, principal component analysis and eigendecomposition -- those are famous examples involving matrix factorization, which most machine learners and data miners have probably used at least once in their life. For practical applications, it is sufficient to know that these factorizations compute a certain kind of embedding into a subspace where the information relevant to the task at hand is maintained. There are built-in functions computing the required decomposition in an efficient manner for all main programming languages. A little bit less known are the general optimization problems whose solutions return the desired decompositions. Even less known is the fact that these optimization problems are transformed into the most widely used clustering objectives by a restriction to binary values of one or more factor matrices. 
%=======================
% Notation
%======================
\section{Notation}
A formal discussion of the state of clustering requires some notation. We visually distinguish between sets, matrices and vectors by displaying sets in calligraphy style $\mathcal{X},\mathcal{Y},\mathcal{Z},\ldots$, matrices in uppercase mode $X,Y,Z,\ldots$ and vectors and constants as lowercase letters $x,y,z$, which we highlight in bold if we want to avoid confusion with constants. We use $\mathbb{R}$ and $\mathbb{R}_+$ to denote the set of real and nonnegative real values. Accordingly, $\mathbb{N}$ denotes the set of natural numbers. Furthermore, we state the set of all $n\times m$-dimensional partition matrices as $\mathbbm{1}^{m\times n}$. That is, $A\in\mathbbm{1}^{m\times n}$ if and only if $A$ is a binary matrix and every row of $A$ has exactly one entry equal to one. 

We write $\mathbf{1}$ to represent a matrix having all elements equal to 1. The dimension of such a matrix can always be inferred from the context. We denote with the matrix $I$ the identity matrix.
We often neglect stating the range of indices if that is clear from the context. We strive for uniformity in the use of indices, such that the range is usually deducible from the particular index. As such, we usually have $1\leq i\leq n, 1\leq j\leq m$ and $1\leq s\leq r$.

We assume that our data is given by the matrix $D\in\mathbb{R}^{m\times n}$. The data represents $m$ points $D_{j\cdot}$ for $1\leq j\leq m$ in the $n$-dimensional feature space. For every point, we denote with $\mathcal{N}_\epsilon(j)=\{l| \|D_{j\cdot}-D_{l\cdot}\|<\epsilon\}$ its $\epsilon$-neighborhood.

Throughout the paper, we often employ the Heaviside step function $\theta_t$ which rounds real to binary values, i.e., $\theta_t(x)=1$ for $x\geq t$  and $\theta_t(x)=0$ otherwise. We abbreviate $\theta_{0.5}$ to $\theta$ and denote with $\theta(X)=(\theta(X_{ji}))_{ji}$ the entry-wise application of $\theta$ to a matrix $X$.
The operator $\circ$ denotes the Hadamard product which multiplies two matrices of same dimensions element-wise. 
We denote matrix norms as $\|\cdot\|$ for the Frobenius norm and $|\cdot|$ for the entry-wise 1-norm. These norms are equivalent for binary matrices $A$ in the sense that $|A|=\|A\|^2$. The Frobenius inner product is defined for matrices $X$ and $Y$ as $\langle X,Y\rangle_F=\tr(X^\top Y)$. For nonnegative matrices $X$ and $Y$, the Frobenius inner product equates the entrywise 1-norm of the (component-wise) Hadamard product (denoted by $\circ$), i.e., $\langle X,Y\rangle_F=|X\circ Y|$. 
Lastly, we remark that $\log$ denotes the natural logarithm. 
%==================================
% Nonnegative Matrix Factorization
%==================================
\section{Nonnegative Matrix Factorization}
The objective of Nonnegative Matrix Factorization (NMF) makes only a slight alteration to renowned factorizations known as Principal Component Analysis (PCA) or truncated SVD, by requiring that the factor matrices are nonnegative. Unfortunately, this seemingly small modification makes NMF an NP-hard problem~\citep{vavasis2009complexity}, opposed to the polynomially solvable SVD. NMF is originally introduced by \cite{paatero1994positive} under the name positive matrix factorization. It gained much attention since the publications from~\citeauthor{lee1999learning}, showing that the nonnegativity constraints and the resulting parts-based explanation of the data empowers the interpretability of the results. Since then, efficient computations of good approximations to the NMF problem are studied.            

A formal task definition of NMF is given as follows: let $D\in\mathbb{R}_+^{m\times n}$ be the nonnegative data matrix and let $r$ be a specified integer, we refer to as rank. The task of NMF is then to recover nonnegative factors $X\in \mathbb{R}_+^{n\times r}$ and $Y\in \mathbb{R}_+^{m\times r}$ such that $YX^T$ approximates $D$. The quality of the approximation is usually measured by means of the Frobenius norm, that is
\begin{equation}
	\min_{X,Y} F(X,Y) = \frac{1}{2}\left\|D-YX^T\right\|^2 \quad \text{s.t. }X\in \mathbb{R}_+^{n\times r}, Y\in \mathbb{R}_+^{m\times r},\label{eq:NMF} \tag{NMF}
\end{equation}
or the Kullback-Leibler divergence. While the former approach is inherently related to clustering using the Euclidean distance of points, the latter is related to topic models such as probabilitistic latent semantic indexing~\citep{gaussier2005relation,ding2006nonnegative}. We will later discuss the particular relationships of approximations with respect to the Frobenius norm and related clustering objectives. 
%----------------------------
% NMF and Clustering
%----------------------------
\subsection{NMF and Clustering} \label{sec:ZS:NMFClus}
Although initially the difference between NMF and clustering was emphasized~\citep{lee1999learning}, further research affirms inherent clustering properties~\citep{li2006relationships}. In this context, columns of $X$ equate cluster centroids and corresponding columns of $Y$ indicate cluster membership tendencies.
An important property of NMF, which is important in the scope of cluster applications, is the near orthogonality of solutions. That is, the columns of the factor matrices $Y$ and $X$ are approximately orthogonal, implying, e.g., for $Y$, that the inner product $\langle Y_{\cdot s},Y_{\cdot t}\rangle$ is close to zero for $s\neq t$.    

The reason why NMF produces near orthogonal factor matrices is due to its objective function. Every matrix product $YX^\top$ of rank $r$ is computable as the sum of $r$ outer products or rank-1 matrices $Y_{\cdot s}X_{\cdot s}^\top$. Thereby, the approximation error returned by the function $F$ is transformable into the sum
\begin{align*}
\|D-\sum_{s=1}^rY_{\cdot s}X_{\cdot s}^\top\|^2 &= \|D\|^2 -2\langle D,\sum_sY_{\cdot s}X_{\cdot s}^\top\rangle +\langle \sum_sY_{\cdot s}X_{\cdot s}^\top,\sum_sY_{\cdot s}X_{\cdot s}^\top\rangle\\
&= \|D\|^2 -2\sum_s\langle D,Y_{\cdot s}X_{\cdot s}^\top\rangle +\sum_s \|Y_{\cdot s}X_{\cdot s}^\top\|^2 + \sum_{s\neq t}\langle Y_{\cdot s}X_{\cdot s}^\top,Y_{\cdot t}X_{\cdot t}^\top\rangle.
\end{align*}
Here, $\langle\cdot,\cdot\rangle$ denotes the Frobenius inner product. We add $(r-1)\|D\|^2$ to the equation above, which does not affect the solution of the minimization with respect to $X$ and $Y$, and obtain therewith a minimization problem which is equivalent to the one from Eq.~\eqref{eq:NMF} 
\begin{align}
\min_{Y,X}\ \sum_{s=1}^r \left( \|D-Y_{\cdot s}X_{\cdot s}^\top\|^2 + \sum_{t\neq s}\langle Y_{\cdot s}X_{\cdot s}^\top,Y_{\cdot t}X_{\cdot t}^\top\rangle\right).\label{eq:NMFOrth}
\end{align}
This formulation of the NMF objective shows that suitable factorizations are the sum of outer products which approximate the data matrix well while having as little overlap as possible with other outer products. The rightmost term $\langle Y_{\cdot s}X_{\cdot s},Y_{\cdot t}X_{\cdot t}\rangle = (X_{\cdot s}^\top X_{\cdot t})(Y_{\cdot s}^\top Y_{\cdot t})$ vanishes if $X_{\cdot s}$ is orthogonal to $X_{\cdot t}$ or $Y_{\cdot s}$ to $Y_{\cdot t}$. Due to the nonnegativity of the matrices, this is the case if the sets of nonzero entries from two distinct columns of a factor matrix do not overlap. 
%-------------------------------
% Algorithms for NMF
%-------------------------------
\subsection{Algorithms for NMF}
There are numerous approaches to solve the optimization problem of NMF in its various versions, including regularizations and constraints. Here, we focus on three approches, which are particularly applied to solve matrix factorizations with binary constraints. For a more thorough overview and discussion of optimization aspects, we refer the reader to~\cite{cichocki2009nonnegative} and~\cite{kim2014algorithms}.
%
% Alternating Minimization
%____________________________
\subsubsection{Alternating Minimization} \label{sec:ZS:AlternatingMinimization}
The function $F$ from Eq.~\eqref{eq:NMF} is nonconvex, but convex in either $X$ or $Y$, if the other argument is fixed. This property ensures that the \emph{Gauss-Seidel} scheme, also known as \emph{block-coordinate descent}, an \emph{alternating minimization} along one of the matrices while the other one is fixed, returns a stationary point upon convergence. Indeed, \cite{grippo2000convergence} have shown that the nonnegativity constraint ensures the convergence of the sequence $(X_k,Y_k)$ generated by the update rule
\begin{align}\label{eq:als}
\begin{split}
X_{k+1} &\in \argmin_{X} F(X,Y_k)\\
Y_{k+1} &\in \argmin_{Y} F(X_{k+1},Y).
\end{split}
\end{align}
The drawback of this method is solving the subproblems in every iteration, which can be computationally intensive. Thus, existing algorithms for NMF approximate the Gauss-Seidel scheme in several ways~\citep{wang2013nonnegative}. Often, the minimization step is replaced by a single gradient descent update. In that respect, the following optimization procedures are all lazy implementations of the alternating minimization.
%
% Multiplicative Updates
%____________________________
\subsubsection{Multiplicative Updates}
NMF received much attention since the publication of the easily implementable multiplicative update algorithm by \cite{lee2001algorithms}. The update rules are defined as element-wise multiplications, given by
\begin{align}
X_{is} &\leftarrow X_{is} \frac{D_{\cdot i}^\top Y_{\cdot s}}{X_{i\cdot}Y^\top Y_{\cdot s}}\label{eq:multUpdateX}\\
Y_{js} &\leftarrow Y_{js}\frac{D_{j\cdot}X_{\cdot s}}{Y_{j\cdot}X^\top X_{\cdot s}}. \label{eq:multUpdateY} 
\end{align}
The multiplication with nonnegative factors preserves the nonnegativity of the matrices. Yet, the above update rules are also transformable into gradient descent updates. The step size is small enough such that the nonnegativity of the factor matrices is preserved.

One of the major advantages of multiplicative updates is its compliance with some constraints, such as nonnegativity or orthogonality of factor matrices. The  convergence to local extremal points is often derivable from the Karush-Kuhn-Tucker conditions. Unsurprisingly, there are many proposals to solve binary constrained matrix factorization by multiplicative updates.  

The major drawback of multiplicative updates is the conservative choice of the step-size, which has to be small enough such that a step into a descent direction does not leave the feasible set. This results in a very slow convergence and makes the approach suitable only for smaller datasets. In addition, from Eqs.~\eqref{eq:multUpdateX} and~\eqref{eq:multUpdateY} follows that every entry in the factor matrices becoming zero is going to stay zero until the end of the optimization. This inflexibility of the optimization scheme makes solutions likely to converge to less optimal minima, since mistakes can not be corrected. 
%
% Proximal Minimization
%____________________________
\subsubsection{Proximal Minimization and Projected Gradient} \label{sec:ZS:ProxNMF}
Instead of restricting the step size to persistently satisfy the nonnegativity constraint, larger step-sizes can be employed and possibly negative entries can be projected to the positive orthant. Using larger step-sizes improves the convergence rate, but the step-size should not be too large, otherwise the iterates might be zig-zagging around the optimum. We can always fall back upon using linesearch such as the Armijo rule~\citep{lin2007projected}, but having the possibility to calculate suitable stepsizes according to a specified strategy is desirable in practice.  

Assuming that we have a good strategy to efficiently determine the step-sizes $\alpha_k$ and $\beta_k$, the projected gradient procedure for nonnegative matrix factorization performs the following updates:
\begin{align*}
X_{k+1}&\gets X_k-\alpha_k\nabla_XF(X_k,Y_k) \\
X_{k+1}&\gets \theta_0(X_{k+1})\circ X_{k+1}\\
Y_{k+1}&\gets Y_k-\beta_k\nabla_YF(X_{k+1},Y_k))\\
Y_{k+1}&\gets \theta_0(Y_{k+1})\circ Y_{k+1}.
\end{align*}
We employ here the Heaviside step function to perform the projection step, the operation $\theta_0(X)\circ X$ sets all nonnegative entries to zero. Projected gradient procedures work well in practice, the crucial aspect is however the determination of the step size in order to ensure convergence. There has been little theory for this optimization scheme until proximal methods have been researched for nonconvex problems~\cite{bolte2014proximal}. We will discuss this theory of proximal methods in detail in Chapter~\ref{chap:PALMB}.

%==============================
% One-Sided Clustering
%==============================
\section{One-Sided Clustering}\label{sec:ZS:onesided}
In a nutshell, hard clustering aims at grouping data points according to a notion of similarity. A fundamental concept of clustering is to find a trade-off between intra-cluster similarity and inter-cluster distance; to minimize distances of points within a cluster while maximizing distances of points from distinct clusters. The definition of the term distance is deciding for the resulting clustering task. 
%Another viewpoint to the minimization of within cluster distances shows a relation to feature variance minimization. We again transfer the within cluster point scatter from Eq.~\eqref{eq:withinCluster} to
%\begin{align}
%\sum_{s=1}^r \frac{1}{2|\mathcal{J}_s|} \sum_{j,l\in \mathcal{J}_s} \|D_{j\cdot}-D_{l\cdot}\|^2 
%&= \sum_{s,i} \frac{1}{2|\mathcal{J}_s|} \sum_{j,l\in \mathcal{J}_s} (D_{ij}-D_{il})^2
%= \sum_{s,i} |\mathcal{J}_s|\,  \sigma_{is}, %\label{eq:clusterVariance}
%\end{align}
%where $\sigma_{is}$ is the sample variance of feature $i$ in cluster $\mathcal{J}_s$. 
%-----------------------------------
\begin{table}%[!hp]
	\centering
    \caption{The matrix factorization objectives of one-sided clustering.}
    \resizebox{\linewidth}{!}{%
	\begin{tabular}{lll}\toprule
  & \multicolumn{2}{c}{Minimize $\|A-ZZ^\top\|^2$ or maximize $\tr(Z^\top A Z)$, subject to $Y\in\mathbbm{1}^{m\times r}$ and} \\\midrule
 $k$-Means & $Z=Y(Y^\top Y)^{-1/2}$ & $A=DD^\top$\\
 Kernel $k$-Means & $Z=Y(Y^\top Y)^{-1/2}$ & $A=K$\\
Normalized Cut & $Z=I_W^{1/2}Y(Y^\top I_W Y)^{-1/2}$ &  $A=-L_s$ or $A=I_W^{-1/2}WI_W^{-1/2}$ \\
Ratio Cut  & $Z=Y(Y^\top Y)^{-1/2}$  &  $A=W-I_W=-L_d$\\
 \bottomrule
\end{tabular}
}
\label{tbl:oneSided}
\end{table}
%-------------------------------------------------
% $k$-Means
%-------------------------------------------------
\subsection{k-Means}
If there is one algorithm which comes to mind when thinking about clustering, it is likely the $k$-means algorithm~\citep{lloyd1982least}. The objective of $k$-means is founded in the intuitively easily graspable notions of clusters by the within-point scatter, minimizing the intra-cluster similarity.
Suppose we are given a partition of $m$ points reflected by the clusters $\mathcal{J}_1,\ldots, \mathcal{J}_r$. A cluster is here the set $\mathcal{J}_s\subseteq \{1,\ldots, m\}$ of its point indices. The average distance of points within a cluster is then given as
\begin{align*}
\frac{1}{2}\sum_{s=1}^r\frac{1}{|\mathcal{J}_s|}\sum_{j,l\in \mathcal{J}_s} \|D_{j\cdot}-D_{l\cdot}\|^2 
&= \sum_{s=1}^r \frac{1}{|\mathcal{J}_s|}\left(\sum_{j\in \mathcal{J}_s} \|D_{j\cdot}\|^2|\mathcal{J}_s| - \sum_{j,l\in \mathcal{J}_s}\langle D_{j\cdot},D_{l\cdot}\rangle\right).\\
&= \sum_{s=1}^r \sum_{j\in \mathcal{J}_s} \|D_{j\cdot }\|^2 -\sum_{j\in \mathcal{J}_s}\left\langle D_{j\cdot},\frac{1}{|\mathcal{J}_s|}\sum_{l\in\mathcal{J}_s}D_{l\cdot}\right\rangle
\end{align*}
We define the matrix $X\in\mathbb{R}^{n\times r}$ by setting the columns $X_{\cdot s}=\frac{1}{|\mathcal{J}_s|}\sum_{l\in\mathcal{J}_s}D_{l\cdot}$ to the centroid of all points in cluster $\mathcal{J}_s$. Therewith, we continue the transformations to
\begin{align}
\frac{1}{2}\sum_{s=1}^r\frac{1}{|\mathcal{J}_s|}\sum_{j,l\in \mathcal{J}_s} \|D_{j\cdot}-D_{l\cdot}\|^2 
&= \sum_{s=1}^r \sum_{j\in \mathcal{J}_s} \|D_{j\cdot }\|^2 -2\sum_{j\in \mathcal{J}_s}\langle D_{j\cdot},X_{\cdot s}\rangle +\sum_{j\in \mathcal{J}_s}\|X_{\cdot s}\|^2 \label{eq:withinCluster}\\
&= \sum_s \sum_{j\in \mathcal{J}_s} \|D_{j\cdot} -X_{\cdot s}^\top\|^2.\nonumber
\end{align}
The equation above is the starting point from which various relationships between clustering and matrix factorization under binary constraints follows. The left term in Eq.~\eqref{eq:withinCluster} is the mentioned within cluster point scatter and it easy to show that a minimization of this term goes along with the maximization of the average distance of points from distinct clusters~\citep{friedman2001elements}. 

We now introduce the binary matrix $Y\in\mathbbm{1}^{m\times r}$ to indicate the cluster partition given by the sets $\mathcal{J}_s$. That is, $Y_{js}=1$ if point $j\in \mathcal{J}_s$ and $Y_{js}=0$ otherwise. The objective of $k$-means is then to find the partition matrix $Y$ minimizing
\begin{equation}
\begin{aligned}
\min_Y &\sum_{s=1}^r\sum_{j=1}^mY_{js}\|D_{j\cdot}-X_{\cdot s}^\top\|^2\qquad
\text{s.t. } & Y\in \mathbbm{1}^{m\times r},\ X_{\cdot s} = \frac{1}{|Y_{\cdot s}|}Y_{\cdot s}^\top D. 
\end{aligned}
\tag{KM}\label{eq:KM}
\end{equation} 
The Problem~\eqref{eq:KM} is transferrable into a constrained nonnegative matrix factorization problem and is likewise NP-hard~\citep{aloise2009np}. Since there is for every point $j$ only one cluster $s$ such that $Y_{js}=1$, we can pull the outer sum into the norm, that is
\begin{align*}
\sum_{s=1}^r\sum_{j=1}^mY_{js}\|D_{j\cdot}-X_{\cdot s}^\top\|^2
&= \sum_{j=1}^m\|D_{j\cdot}-\sum_{s=1}^rY_{js}X_{\cdot s}^\top\|^2
= \sum_{j=1}^m\|D_{j\cdot}-Y_{j\cdot}X^\top\|^2\\
&=\|D-YX^\top\|^2.
\end{align*}
The cluster centers are in matrix notation given by the product $X=D^\top Y^\top (Y^\top Y)^{-1}$. 
The matrix $Y^\top Y=\diag(|Y_{\cdot 1}|,\ldots, |Y_{\cdot r}|)$ is diagonal, since $Y$ has orthogonal columns. Thus, the inverse is easily computed by inverting the elements on the diagonal. Note, that the matrix $(Y^\top Y)^{-1}Y^\top=Y^\dagger$ is the Moore-Penrose inverse of the matrix $Y$.

The clustering of $k$-means has multiple equivalent formulations, which we summarize in the following theorem, whose formal proof is provided in Appendix~\ref{chap:AppendixZS}.
%----------- thm kmeans obj---------
\begin{restatable}{theorem}{kmeansobj}\label{thm:kmeansobj}
The following optimization problems are equivalent to Problem~\eqref{eq:KM}
\begin{align}
\min_{Y}\ &\|D-YX^\top\|^2 &\text{ s.t. } Y\in\mathbbm{1}^{m\times r}, X=D^\top Y(Y^\top Y)^{-1} \label{eq:kmeansYXmean} \\
\min_{Y}\ &\|D-YY^\dagger D\|^2 &\text{ s.t. } Y\in\mathbbm{1}^{m\times r} \label{eq:kmeansYYdaggerD} \\
\min_{Y,X}\ &\|D-YX^\top\|^2 &\text{ s.t. } Y\in\mathbbm{1}^{m\times r}, X\in\mathbb{R}^{n\times r} \label{eq:kmeansYX} \\
\max_{Y}\ &\tr(Z^\top DD^\top Z)&\text{ s.t. } Z= Y(Y^\top Y)^{-1/2}, Y\in\mathbbm{1}^{m\times r} \label{eq:kmeansTr} \\
\min_{Y}\ &\|DD^\top - YY^\dagger\|^2 &\text{ s.t. }   Y\in\mathbbm{1}^{m\times r} \label{eq:kmeansDD}
\end{align}
\end{restatable}
The various formulations of the objective of $k$-means give rise to multiple optimization approaches. The equivalence of Eqs.~\eqref{eq:kmeansYXmean} and~\eqref{eq:kmeansYX} establishes an alternating minimization scheme, since the optimal cluster center matrix is provided in closed from, given the cluster assignment matrix. We will discuss this procedure, known als Lloyds' minimization, more in detail in Section~\ref{sec:ZS:AlternatingMinimization}. Since $Y$ is orthogonal and nonnegative, Eq.~\eqref{eq:kmeansYX} is approximable by an orthogonal relaxation (cf. \@Sec.\ref{sec:ZS:OrthogonalRelaxation}).
The objective in Eq.~\eqref{eq:kmeansTr} resembles the optimization task to find the largest eigenvectors of a symmetric real-valued matrix such as $DD^\top$ (cf.\@ Section~\ref{sec:ZS:SpectralRelaxation}). This relation founds the application of a spectral relaxation~\citep{zha2002spectral}. 
Eqs.~\@\eqref{eq:kmeansTr} and~\@\eqref{eq:kmeansDD} show that $k$-means can be formulated in sole dependence of the similarity between points, measured by the inner product which equates the cosine similarity between points when the data points are normalized. This is a stepping stone to the application of kernel methods, discussed in the following section.

The restriction not only to a binary but a partition matrix $Y$ in $k$-means clustering has many favorable outcomes concerning its optimization. An efficient alternating minimization scheme for the more general case where $Y$ is a binary matrix is not known.   However, some applications require more flexible cluster models, allowing for overlap between clusters and outlier detection. Examples for such areas are text mining and gene analysis. A single gene is typically involved in multiple functions of an organism and hence, it should be assignable to multiple clusters. A similar argument holds for documents, addressing more than one topic. \cite{whang2018non} propose a semidefinite program to allow for a specified amount of overlap and a specified amound of outliers. Unfortunately, the resulting semidefinite program has a large amount of constraints and its optimization is quite slow.
\cite{slawski2013matrix} study properties of exact decompositions of the form $D= YX^\top$ where $Y\in\{0,1\}^{m\times r}$ and $X\in\mathbb{R}^{n\times r}$. Based on the observation that at most $2^r$ binary vectors lie in the subspace spanned by the columns of $D$, a combinatorial algorithm is proposed whose complexity is exponential in the rank. This algorithm is extended to find approximate factorizations $D\approx YX^\top$, yet the results display a high variance. 
%--------------------------
% Kernel k-means
%--------------------------
\subsection{Kernel k-Means}
The formulation of the $k$-means objective in sole dependence on the similarities of data points, expressed by the inner product enables the application of kernel methods and the derivation of nonconvex clusters. One of the drawbacks of $k$-means clustering is that it computes a Voronoi tesselation which determines the cluster membership. That is, the allocated regions of adjacent clusters are separated by a line. This entails that clusters which are returned by $k$-means are always convex. 

If the cluster regions are not linearly separable, then a transformation into a suitable, usually higher-dimensional space enables a correct identification of clusters by $k$-means. Let $\Phi:\mathbb{R}^n\rightarrow \mathcal{H}$ be such a transformation from the feature space to a (possibly infinite-dimensional) Hilbert space with inner product $\langle\cdot,\cdot\rangle_\mathcal{H}$. The similarities between two points, which are reflected by the inner product in Eq.~\eqref{eq:kmeansDD} are given in the transformed space by the kernel matrix $K\in\mathbb{R}^{m\times m}$, where   
\begin{align}
K_{jl} = \left\langle\Phi(D_{j\cdot}^\top),\Phi(D_{l\cdot}^\top)\right\rangle_\mathcal{H}. \label{eq:kernel}
\end{align}
Since the features are only transformed within the inner product, we can employ the \emph{kernel trick} by computing the inner product in the transformed space directly. This application has been proposed by~\cite{scholkopf1998nonlinear} in the more general scope of truncated SVD or PCA. \cite{ding2005equivalence} discuss the application of kernels in the scope of $k$-means. 
We conclude the following more generally formulated corollary from Theorem~\ref{thm:kmeansobj}, stated for symmetric matrices.
\begin{corollary}\label{thm:kkmeansobj}
Let $K\in\mathbb{R}^{m\times m}$ be a symmetric matrix and let $K=UU^\top$ be a symmetric decomposition of $K$ with $U\in\mathbb{R}^{m\times m}$. The following optimization problems are equivalent:
\begin{align}
&\max_{Y}\ \tr(Z^\top K Z)  & \text{s.t. } Z=Y(Y^\top Y)^{-1/2}, Y\in\mathbbm{1}^{m\times r}\label{eq:kkMeanstr}\\
&\min_{Y}\ \|K - YY^\dagger\|^2 & \text{s.t. }  Y\in\mathbbm{1}^{m\times r}\label{eq:kkMeansnorm}\\
&\min_{Y,X}\ \|U-YX^\top\|^2 & \text{s.t. } Y\in\mathbbm{1}^{m\times r}, X\in\mathbb{R}^{m\times r} \label{eq:kkmeansU}
\end{align}
\end{corollary}
We note that Eq.~\eqref{eq:kkmeansU} states the $k$-means objective on the matrix $U$. This follows directly from substituting $K$ with a symmetric decomposition $UU^\top=K$ in Eq.~\eqref{eq:kkMeanstr} and Theorem~\ref{thm:kmeansobj}. The existence of such a decomposition is guaranteed because $K$ is a symmetric and real-valued matrix. Possible decomposition methods are inter alia the Cholesky factorization, where $U$ is an upper triangular matrix, or the eigendecomposition, where the columns of $U$ return scaled eigenvectors of $K$.    
%----------------------------------
% Graph Cut 
%----------------------------------
\subsection{Graph Cuts and Factorizing Graph Laplacians} \label{sec:ZS:graphCut}
The objective of kernel $k$-means in sole dependence of similarities between points introduces another data representation by means of a graph. The symmetric, real-valued kernel matrix has also an interpretation as a weighted adjacency matrix $W$ to a graph $\mathcal{G}=(\mathcal{V},\mathcal{E})$. Every data point $D_{j\cdot}$ corresponds thereby to a node and the edges $\{j,l\}\in\mathcal{E}$ are defined by those entries where $W_{jl}$ is larger than zero. The larger the weight of an edge, the stronger the connection between its nodes. Now, let us transfer the notion of clusters based on maximizing the inter-cluster similarity to graphs. A cluster is then a set of nodes having weak connections to nodes outside of the cluster. In other words, if  we imagine to cut the edges connecting every cluster to its outside, then we strive to cut as few (strong) edges as possible. However, this formulation entails that single nodes, which are not connected, are optimal cluster candidates. In order to favor larger clusters, \cite{hagen1992new} propose to normalize the cut weights with the cluster size, introducing the rational cut objective
\begin{align*}
\min_Y RCut(Y;W) = \sum_{s=1}^r \frac{Y_{\cdot s}^\top W(\mathbf{1}-Y_{\cdot s})}{|Y_{\cdot s}|} \quad \text{s.t. } Y\in \mathbbm{1}^{m\times r}.
\end{align*}
The numerator $Y_{\cdot s}^\top W(\mathbf{1}-Y_{\cdot s})$ returns the cut from the nodes inside to the nodes outside of the cluster $s$.
\cite{shi2000normalized} incorporate the edge weights into the normalization term. This modification allows for larger cut weights of strongly connected clusters. The objective is called normalized cut, given as
\begin{align*}
\min_Y NCut(Y;W) = \sum_{s=1}^r \frac{Y_{\cdot s}^\top W(\mathbf{1}-Y_{\cdot s})}{Y_{\cdot s}I_WY_{\cdot s}} \quad \text{s.t. } Y\in \mathbbm{1}^{m\times r}.
\end{align*}
Both objectives, rational and normalized cut, are transferable into a form which is similar to the kernel $k$-means objective. We transform the cut value of cluster $s$ to
\begin{align*}
	Y_{\cdot s}^\top W(\mathbf{1}-Y_{\cdot s})=\sum_{j=1}^m Y_{js}(|W_{j\cdot}|-W_{j\cdot}Y_{\cdot s}) =  \sum_{j=1}^m Y_{js}(|W_{j\cdot}|Y_{js}-W_{j\cdot}Y_{\cdot s}) = Y_{\cdot s}^\top (I_W-W)Y_{\cdot s}.
\end{align*}
Using this transformation, the objective cut functions correspond to a trace optimization problem of the form
\begin{align*}
\min_Y &\sum_{s=1}^r\frac{Y_{\cdot s}^\top W (\mathbf{1}-Y_{\cdot s})}{Y_{\cdot s}^\top J Y_{\cdot s}} 
&= \tr\left(Y^\top (I_{W}-W)Y (Y^\top JY)^{-1}\right) &\text{s.t. }Y\in\mathbbm{1}^{m\times r},
\end{align*}
where $J$ is a diagonal matrix. If we set $J=I$ then the objective function above is equal to rational cut and $J=I_W$ corresponds to normalized cut.
We observe that the trace optimization problem is similar to the kernel $k$-means objective~\eqref{eq:kkMeanstr} from Theorem~\ref{thm:kkmeansobj}. Correspondingly, we derive the following equivalence of optimization problems.
%---- COROLLARY MINCUT TRACE MAXIMIZATION ------------
\begin{corollary}\label{thm:cut}
Let $W\in\mathbb{R}^{m\times m}$ be a symmetric, real valued matrix and let $J\in\{I,I_W\}$ be an $m\times m$ diagonal matrix. The matrix 
%\begin{align*}
$L=J^{-1}I_W-J^{-1/2}WJ^{-1/2}$
%\end{align*}
is positive semidefinite, having eigenvalues $0=\lambda_1\leq\ldots\leq \lambda_m$. Denote with $-L+\lambda_mI=UU^\top$ a symmetric decomposition, then the following optimization problems are equivalent: 
\begin{align}
&\min_Y\ \sum_{s=1}^r\frac{Y_{\cdot s}^\top W (\mathbf{1}-Y_{\cdot s})}{Y_{\cdot s}^\top J Y_{\cdot s}} & \text{s.t. } Y\in\mathbbm{1}^{m\times r},\\
&\min_Y\ \tr\left(Z^\top LZ\right), & \text{s.t. } Z=J^{1/2}Y(Y^\top JY)^{-1/2}, Y\in\mathbbm{1}^{m\times r},\label{eq:minCutTr}\\ 
&\min_Y \|-L-ZZ^\top\|^2, & \text{s.t. } Z=J^{1/2}Y(Y^\top JY)^{-1/2}, Y\in\mathbbm{1}^{m\times r}, \label{eq:minCutNorm} \\
&\min_Y \|UU^\top-ZZ^\top\|^2 & \text{s.t. } Z=J^{1/2}Y(Y^\top JY)^{-1/2}, Y\in\mathbbm{1}^{m\times r}, \label{eq:minCutU}
\end{align}
\end{corollary}
%---------------------------------------
The name spectral clustering derives from its optimization scheme using a spectral relaxation (cf. \@Section~\ref{sec:ZS:SpectralRelaxation}), based on Eq.~\eqref{eq:minCutTr}.
The matrix $L$ from Corollary~\ref{thm:cut} for $J\in\{I,I_W\}$ is called graph Laplacian~\citep{mohar1991laplacian,chung1997spectral}. 
%--- TABLE: LAPLACIANS -------------
\begin{table}%[!hp] 
	\centering
    \caption{Popular graph Laplacians and corresponding eigenproblems, returning the same set of eigenvectors $v$ and eigenvectors $\lambda$.}
    %\resizebox{\linewidth}{!}
\label{tbl:laplacians}
\end{table}
%------------------------------------
There three popular definitions of a graph Laplacian, which we summarize in Table~\ref{tbl:laplacians} together with the relations between their spectra.   
The spectrum of graph Laplacians is interesting because the multiplicity of their smallest eigenvalue, which is zero, is equal to the number of connected components in the graph. Furthermore, the eigenvectors to the eigenvalue zero indicate precisely the connected components.
This property is easily understood, considering that suitably reordering the columns and rows of $W$ results in a block-diagonal form of $W$. Let us consider a single connected component $\mathcal{C}\subseteq\mathcal{V}$, consisting of $d$ nodes. If $W$ is arranged such that the nodes of the connected component are reflected by the first $d$ columns and rows, then every edge weight $W_{jl}$ with $j\in\mathcal{C}$ and $l\notin\mathcal{C}$ is equal to zero; otherwise $l$ would belong to the connected component, Thus, $W$ appears as a block diagonal matrix. Let $v\in\{0,1\}^m$ be the vector indicating the set $\mathcal{C}$, having the first $d$ entries equal to one and all other entries equal zero. Then $v$ satisfies the following relation:
\[
Wv = 
\begin{tikzpicture}[baseline=-.65ex]
\matrix[
  matrix of math nodes, left delimiter=(, right delimiter=),
  column sep=1ex,
] (m)
{
 W_{11}&\ldots&W_{1d} & \\
 \vdots&&\vdots& \mathbf{0}\\
 W_{d1}&\ldots&W_{dd}& \\
 & \mathbf{0} & & \widehat{W} \\
};
\draw[dashed] ([xshift=0.5ex]m-1-3.north east) -- ([xshift=0ex]m-4-4.south west);
\draw[dashed] (m-3-1.south west) -- (m-4-4.north east);
\end{tikzpicture}
\begin{tikzpicture}[baseline=-.65ex]
\matrix[
  matrix of math nodes, left delimiter=(, right delimiter=),
  column sep=1ex,
] (mv)
{
 1\\
 \vdots\\
 1 \\
 \mathbf{0} \\
};
\draw[dashed] (mv-3-1.south west) -- (mv-3-1.south east);
\end{tikzpicture}
=
\begin{tikzpicture}[baseline=-.65ex]
\matrix[
  matrix of math nodes, left delimiter=(, right delimiter=),
  column sep=1ex,
] (mr)
{
 \mid W_{1\cdot}\mid\\
 \vdots\\
 \mid W_{d\cdot}\mid \\
 \mathbf{0} \\
};
\draw[dashed] (mr-3-1.south west) -- (mr-3-1.south east);
\end{tikzpicture}
=I_Wv.
\]
Subtracting $I_Wv$ in the equation above yields that $v$ is an eigenvector of the difference Laplacian with eigenvalue zero. Similarly, multiplying with $I_W^{-1}$ from the left, yields that $v$ is an eigenvector of the random walk Laplacian and multiplying with $I_W^{-1/2}$ from the left shows that $I_W^{1/2}v$ is an eigenvector of the symmetric Laplacian with eigenvalue zero. 

In the best but unlikely case, the connected components reflect exactly the desired clustering. However in general, the graph representation $W$ is build in a preprocessing step by a gauss kernel matrix, the $\epsilon$-neighborhood or $k$-nearest neighbor graph. Since the resulting edge weights depend on the employed method and parameter setting of this preprocessing step, we can generally not assume that the graph exhibits perfect clustering properties. Therefore, newer approaches of spectral clustering aim at learning the graph representation together with the resulting clustering~\cite{bojchevski2017robust,kang2018unified}

%==============================
% Two-Sided Clustering
%==============================
\section{Two-Sided Clustering}\label{sec:ZS:TwoSided}
In some application areas, such as collaborative filtering and gene expression analysis, we can not expect that clusters are identifiable based on the similarity of data points on the whole feature space. As an example, we cite the clustering of users according to movie preferences. Given a user times movie database, where each entry reflects the given rating, we most likely will not be able to find a set of users which give similar ratings on all movies. This effect can be seen as a manifestation of the \emph{curse of dimensionality}, broadly stating that points in a high dimensional data set are likely to be approximately equidistant~\citep{aggarwal2001surprising,beyer1999nearest}. That aside, we will likely be able to identify a group of users together with a small subset of the provided movies where the ratings are similar. This introduces the task of subspace clustering, the identification of the subspaces in which data points exhibit a clear cluster structure~\citep{kriegel2009clustering}.   

Assuming that the cluster subspaces are spanned by a subset of the features, matrix factorizations for subspace clustering require not one but two binary matrices; one on the left to indicate the clustering of the data points and one on the right to indicate the feature space in which the data points cluster. If the data matrix is real-valued, then a third middle matrix is employed to scale the indicated clusters. In this case, we speak of a tri-factorization.  
\begin{figure}
    \centering
    \begin{align*}
\begin{tikzpicture}[baseline=-0.5ex,
    style1/.style={
  matrix of math nodes,
  every node/.append style={text width=#1,align=center,minimum height=5ex},
  nodes in empty cells,
  left delimiter=(,
  right delimiter=),ampersand replacement=\&}
  ]
   \matrix [style1=5ex] (n) {
C_{11} \& C_{12} \& C_{13}\\
C_{21} \& C_{22} \& C_{23}\\
};
\draw[color=col1,fill=col1,opacity=0.4] (n-1-1.north west) -- (n-1-1.north east) -- (n-1-1.south east) -- (n-1-1.south west) -- (n-1-1.north west);
\draw[color=col5,fill=col5,opacity=0.4] (n-1-2.north west) -- (n-1-2.north east) -- (n-1-2.south east) -- (n-1-2.south west) --(n-1-2.north west);
\draw[color=col2,fill=col2,opacity=0.4] (n-1-3.north west) -- (n-1-3.north east) -- (n-1-3.south east) -- (n-1-3.south west) --(n-1-3.north west);
\draw[color=col4,fill=col4,opacity=0.4] (n-2-1.north west) -- (n-2-1.north east) -- (n-2-1.south east) -- (n-2-1.south west) --(n-2-1.north west);
\draw[color=col3,fill=col3,opacity=0.4] (n-2-2.north west) -- (n-2-2.north east) -- (n-2-2.south east) -- (n-2-2.south west) --(n-2-2.north west);
\draw[color=col6,fill=col6,opacity=0.4] (n-2-3.north west) -- (n-2-3.north east) -- (n-2-3.south east) -- (n-2-3.south west) --(n-2-3.north west);
\end{tikzpicture}
& \quad
\begin{tikzpicture}[baseline=-0.5ex,
    style1/.style={
  matrix of math nodes,
  every node/.append style={align=center,minimum height=1.6ex,minimum width=1.6ex},
  nodes in empty cells,
  left delimiter=(,
  right delimiter=),ampersand replacement=\&}
  ]
   \matrix [style1] (m) {
 \& \& \& \& \& \& \& \& \& \\
 \& \& \& \& \& \& \& \& \&\\
 \& \& \& \& \& \& \& \& \&\\
 \& \& \& \& \& \& \& \& \&\\
 \& \& \& \& \& \& \& \& \&\\
 \& \& \& \& \& \& \& \& \& \\
};
%first 3 vertical
\draw[color=col1,fill=col1,opacity=0.6] (m-1-1.north west) -- (m-1-1.north east) -- (m-6-1.south east) -- (m-6-1.south west) -- (m-1-1.north west);
\draw[color=col1!80,fill=col1!80,opacity=0.6] (m-1-2.north west) -- (m-1-2.north east) -- (m-6-2.south east) -- (m-6-2.south west) --(m-1-2.north west);
\draw[color=col1!60,fill=col1!60,opacity=0.6] (m-1-3.north west) -- (m-1-3.north east) -- (m-6-3.south east) -- (m-6-3.south west) --(m-1-3.north west);
%second 3 vertical
%\draw[color=col3,fill=col3,opacity=0.6] (m-1-4.north west) -- (m-1-4.north east) -- (m-6-4.south east) -- (m-6-4.south west) -- (m-1-4.north west);
%\draw[color=col3!80,fill=col3!80,opacity=0.6] (m-1-5.north west) -- (m-1-5.north east) -- (m-6-5.south east) -- (m-6-5.south west) --(m-1-5.north west);
%\draw[color=col3!60,fill=col3!60,opacity=0.6] (m-1-6.north west) -- (m-1-6.north east) -- (m-6-6.south east) -- (m-6-6.south west) --(m-1-6.north west);
%last 3 vertical
\draw[color=col5,fill=col5,opacity=0.6] (m-1-8.north west) -- (m-1-8.north east) -- (m-6-8.south east) -- (m-6-8.south west) -- (m-1-8.north west);
\draw[color=col5!80,fill=col5!80,opacity=0.6] (m-1-9.north west) -- (m-1-9.north east) -- (m-6-9.south east) -- (m-6-9.south west) --(m-1-9.north west);
\draw[color=col5!60,fill=col5!60,opacity=0.6] (m-1-10.north west) -- (m-1-10.north east) -- (m-6-10.south east) -- (m-6-10.south west) --(m-1-10.north west);
%first 3 horizontal
\draw[color=col2,fill=col2,opacity=0.3] (m-1-1.north west) -- (m-1-10.north east) -- (m-1-10.south east) -- (m-1-1.south west) -- (m-1-1.north west);
\draw[color=col2,fill=col2,opacity=0.2] (m-2-1.north west) -- (m-2-10.north east) -- (m-2-10.south east) -- (m-2-1.south west) --(m-2-1.north west);
%\draw[color=col2,fill=col2,opacity=0.1] (m-3-1.north west) -- (m-3-10.north east) -- (m-3-10.south east) -- (m-3-1.south west) --(m-3-1.north west);
%second 3 horizontal
\draw[color=col6,fill=col6,opacity=0.3] (m-4-1.north west) -- (m-4-10.north east) -- (m-4-10.south east) -- (m-4-1.south west) -- (m-4-1.north west);
\draw[color=col6,fill=col6,opacity=0.2] (m-5-1.north west) -- (m-5-10.north east) -- (m-5-10.south east) -- (m-5-1.south west) --(m-5-1.north west);
\draw[color=col6,fill=col6,opacity=0.1] (m-6-1.north west) -- (m-6-10.north east) -- (m-6-10.south east) -- (m-6-1.south west) --(m-6-1.north west);
\end{tikzpicture}
\quad
%---------- BLOCKDIAGONAL----------------
\begin{tikzpicture}[baseline=-0.5ex,
    style1/.style={
  matrix of math nodes,
  every node/.append style={text width=#1,align=center,minimum height=5ex},
  nodes in empty cells,
  left delimiter=(,
  right delimiter=),ampersand replacement=\&}
  ]
   \matrix [style1=5ex] (n) {
C_{11} \& \& \\
 \& C_{22} \& \phantom{C_{22}} \\
};
\draw[color=col1,fill=col1,opacity=0.4] (n-1-1.north west) -- (n-1-1.north east) -- (n-1-1.south east) -- (n-1-1.south west) -- (n-1-1.north west);
\draw[color=col3,fill=col3,opacity=0.4] (n-2-2.north west) -- (n-2-3.north east) -- (n-2-3.south east) -- (n-2-2.south west) --(n-2-2.north west);
\end{tikzpicture}
\end{align*}
    \caption{Variants of row- and column- partitioning biclusters: checkerboard model (left), plaid model (right). Best viewed in color. }
    \label{fig:blockclusters}
\end{figure}
%--- TABLE: TWO-Sided Clustering -------------
\begin{table}%[!hp] 
	\centering
    \caption{Overview of objective functions proposed for two-sided clustering}
    \resizebox{\linewidth}{!}{%
	\begin{tabular}{llr}\toprule
Decomposition &  Objective &  \\ \midrule
Checkerboard  & $\displaystyle\min_{X,C,Y}\|D-YCX^\top\|^2$ & s.t. $X\in\mathbbm{1}^{n\times r},Y\in\mathbbm{1}^{m\times \hat{r}},C\in\mathbb{R}_+^{r\times \hat{r}}$\\
Plaid  & \multicolumn{2}{l}{$\displaystyle\min_{X,C,Y}\|D-YY^\dagger D - D XX^\dagger + YCX^\dagger\|^2$ s.t. $X\in\mathbbm{1}^{n\times r},Y\in\mathbbm{1}^{m\times \hat{r}},C=Y^\dagger D X$}\\
Diagonal & $\displaystyle\min_{X,C,Y}\|D-YCX^\top\|^2$ & s.t. $X\in\mathbbm{1}^{n\times r},Y\in\mathbbm{1}^{m\times r},C=\diag(C_{11},\ldots,C_rr)$ \\
Binary & $\displaystyle\min_{X,Y}\|D-YX^\top\|^2$ & s.t. $X\in\{0,1\}^{n\times r},Y\in\{0,1\}^{m\times r}$\\
Boolean & $\displaystyle\min_{X,Y}\|D-Y\odot X^\top\|^2$ & s.t. $X\in\{0,1\}^{n\times r},Y\in\{0,1\}^{m\times r}$\\
 \bottomrule
\end{tabular}
}
\label{tbl:twoSidedClustering}
\end{table}
We summarize the discussed factorizations of this section in Table~\ref{tbl:twoSidedClustering}. The discussed models are also known as two-mode clusters~\citep{van2004two,van2009optimization} or biclusters›\citep{busygin2008biclustering}.
%-----------------------------------
% Checkerboard Clustering
%-----------------------------------
\subsection{Checkerboard Clustering}
The model of checkerboard clustering assumes that the data matrix is partitioned into a set of row clusters $\mathcal{J}_1,\ldots,\mathcal{J}_q\subseteq\{1,\ldots,m\}$ and a set of column clusters $\mathcal{I}_1,\ldots,\mathcal{I}_{r}\subseteq\{1,\ldots,n\}$ such that every combination of a row- with a column-cluster creates a bicluster $(\mathcal{I}_s,\mathcal{J}_t)$. Note, that the numbers of row and column clusters are potentially different. We denote here with the bicluster set $\mathcal{B}$ the collection of all possible combinations of row- and column-clusters. 
The elements of the data matrix belonging to a bicluster are approximated with the mean value in the bicluster. Hence, the task of checkerboard clustering is to find an optimal partition of rows and columns such that every data entry does not differ much from the average data value in the bicluster. More formally, the objective is as follows:
\begin{align}
   \min_{\mathcal{B}} \sum_{i=1}^n\sum_{j=1}^m \left(D_{ji}-\sum_{ (\mathcal{J},\mathcal{I})\in\mathcal{B}:\  (i,j)\in\mathcal{J}\times\mathcal{I}}\mu(D_{\mathcal{JI}})\right)^2 \label{eq:blockConst}
\end{align}
This objective is transferable into a matrix factorization problem, involving a tri-factorization as outlined by the following theorem.

\begin{theorem}\label{thm:checkerboard}
The following optimization problems are equivalent to the objective from Eq.~\eqref{eq:blockConst} subject to
\[\mathcal{B}\in \left\{\{\mathcal{I}_1,\ldots,\mathcal{I}_r\}\times \{\mathcal{J}_1,\ldots,\mathcal{J}_q\}\mid \dot\bigcup_{s}\mathcal{I}_s=\{1,\ldots,n\},\  \dot\bigcup_{t}\mathcal{J}_t=\{1,\ldots,m\}\right\},\]
that is the set of biclusters is given as the cartesian product of a partition of the rows and the columns.
\begin{align}
    \label{eq:checkerOrig}
    \min_{X,C,Y}&\ \|D-YCX^\top\|^2 &
    \text{s.t. } X\in \mathbbm{1}^{n\times r},\ Y\in\mathbbm{1}^{m\times r}, C\in\mathbb{R}^{\hat{r}\times r}\\ 
    \label{eq:checkerAvg}
\min_{X,Y}&\ \|D-YCX^\top\|^2&
\text{s.t. } X\in \mathbbm{1}^{n\times r},\ Y\in\mathbbm{1}^{m\times r}, C= Y^\dagger D {X^\dagger}^\top\\
\max_{X,Y}&\ \tr(D^\top YCX^\top) &
\text{s.t. } X\in \mathbbm{1}^{n\times r},\ Y\in\mathbbm{1}^{m\times r}, C= Y^\dagger D {X^\dagger}^\top
\end{align}
\end{theorem}
Since the columns in $Y$ and $X$ are orthogonal, there is a permutation of rows and columns such that every bicluster appears as a coherent block in the factorization matrix. This is where the name checkerboard clustering comes from, the visualization by the left matrix in Fig.~\ref{fig:blockclusters} shows that the partition of the data matrix by biclusters results in a checkerboard pattern. There are two row clusters and three column clusters. Each intersection of row cluster $t$ and column cluster $s$ reflects one bicluster, which is approximated by the constant $C_{ts}$. Such an approximation with constant biclusters goes back to ~\cite{hartigan1972direct}, who proposes said approximation for a tree partitioning model.

The equivalence of the objectives from Eqs.~\eqref{eq:checkerOrig} and~\eqref{eq:checkerAvg}  goes back to~\cite{gaul1996new} and allows for an alternating minimization with respect to one matrix, while fixing the other two~\citep{maurizio2001double,wang2011fast,cho2004minimum}. This procedure adapts Lloyd's algorithm from $k$-means clustering (cf.\@ Section~\ref{sec:ZS:Lloyds}). Furthermore, the orthogonal nonnegative relaxation (cf. \@Section~\ref{sec:ZS:OrthogonalRelaxation}) is also adapted for the optimization of tri-factorizations~\citep{ding2006orthogonal, yoo2010orthogonal}. The relationship between checkerboard and $k$-means clustering becomes apparent when determining that every feature belongs to its own cluster, that is setting $X=I_n$ in Eq.~\eqref{eq:checkerAvg}. This modification transfers the two-sided clustering of the checkerboard model into the one-sided clustering of $k$-means. 
%-------------------------------------------
% The Plaid Model
%------------------------------------------
\subsection{The Plaid Model}
The plaid model originates from a bioinformatics application in microarray data analysis. Microarrays are used to measure and reflect the gene expressions of patients. Let us say, the number of patients is $m$ and the number of genes is $n$. Among a set of patients, some genes may co-regulate. That is, the genes exhibit similar expression patterns among the set of patients. Such a set of patients together with the co-regulating genes identifies a bicluster. In this case, saying that every bicluster approximates the corresponding part of the data matrix by a single aggregated value, as known from checkerboard clustering is not enough. Some genes have generally higher or lower expression levels than other genes and the same holds for patients. Therefore, the plaid model introduces bias terms for each patient $\mu(D_{j\mathcal{I}})$ and gene $\mu(D_{\mathcal{J}i})$ to model deviations from the average bicluster value. Denoting again with $\mathcal{B}$ the set of all possible biclusterings, we state the objective of plaid clustering as:
\begin{align}
   \min_{\mathcal{B}} \sum_{i=1}^n\sum_{j=1}^m \left(D_{ji}-\sum_{\substack{(\mathcal{J},\cdot)\in\mathcal{B}:\\  j\in\mathcal{J}}}\mu(D_{\mathcal{J}i}) - \sum_{\substack{(\cdot,\mathcal{I})\in\mathcal{B}:\\  i\in\mathcal{I}}} \mu(D_{j\mathcal{I}})+ \sum_{\substack{(\mathcal{J},\mathcal{I})\in\mathcal{B}:\\  (i,j)\in\mathcal{J}\times\mathcal{I}}} \mu(D_{\mathcal{JI}})\right)^2. \label{eq:plaid}
\end{align}
The objective combines one-sided clusterings of the data matrix and its transposed, resulting in a plaid structure as depicted on the middle of Figure~\ref{fig:blockclusters}. Adding the last term in Eq.~\eqref{eq:plaid} is required as a result of the inclusion-exclusion principle,since the approximation of the data in the intersecting area is subtracted twice, once for row- and once for column- clusters. 

Genes are not expected to take part in only one biological process. Hence, the biclusters are for biological applications generally not restricted to partitions of rows and columns. The algorithms proposed by \cite{cheng2000biclustering, lazzeroni2002plaid} and \cite{turner2005improved} sequentially optimize the biclusters one-by-one. We call this the greedy approach, which is discussed in Section~\ref{sec:ZS:GreedyApproach}. However, assuming the set of biclusters to be restricted to row- and column partitions enables the adaptation of $k$-means optimization procedures. We state here the plaid optimization task with respect to partitioning biclusters in matrix factorization form.
\begin{theorem}
The following optimization problems are equivalent to the plaid optimization problem from Eq.~\eqref{eq:plaid} subject to 
\[\mathcal{B}\in \left\{\{\mathcal{I}_1,\ldots,\mathcal{I}_q\}\times \{\mathcal{J}_1,\ldots,\mathcal{J}_r\}\mid \dot\bigcup_{s}\mathcal{I}_s=\{1,\ldots,n\},\  \dot\bigcup_{t}\mathcal{J}_t=\{1,\ldots,m\}\right\}.\]
\begin{align}
    \min_{X,Y}& \|D-YY^\dagger D - DXX^\dagger+&YCX^\top\|^2  \text{ s.t. } C=Y^\dagger D{X^\dagger}^\top X\in \mathbbm{1}^{n\times r},\ Y\in\mathbbm{1}^{m\times r}\label{eq:plaidFnorm}\\
    \min_{X,Y}&\ \|A-YY^\dagger D\|^2  &\text{s.t. } A=D-DXX^\dagger, X\in \mathbbm{1}^{n\times r},\ Y\in\mathbbm{1}^{m\times r}\label{eq:plaidY}\\
    \min_{X,Y}&\ \|A- AXX^\dagger\|^2  & \text{s.t. } A=D-YY^\dagger A, X\in \mathbbm{1}^{n\times r},\ Y\in\mathbbm{1}^{m\times r}\label{eq:plaidX}
    %\max_{X,Y}&\ \tr(YY^\dagger DD^\top) +\tr(D^\top DXX^\dagger) -\tr(D^\top YY^\dagger D {(XX^\dagger)}^\top)
    %&\text{s.t. } X\in \mathbbm{1}^{n\times r},\ Y\in\mathbbm{1}^{m\times r}
\end{align}
\end{theorem}
\cite{cho2004minimum} proposes the alternating optimizations based on the equality of Eqs.~\eqref{eq:plaidFnorm}, \eqref{eq:plaidY} and \eqref{eq:plaidX}. We observe that the last two objectives are equivalent to the $k$-means clustering given in Eq.~\eqref{eq:kmeansYYdaggerD}. Correspondingly, an optimization performing alternating updates with respect to $Y$ based on Objective~\eqref{eq:plaidY} and with respect to $X$ based on Objective~\eqref{eq:plaidX} converges to local minima of the plaid optimization problem.
%------------------------------------------------
% The Block Diagonal Model
%------------------------------------------------
\subsection{The Block Diagonal Model and Bipartite Graph Cuts}
The models discussed so far allow for varying numbers of row- and column- clusters, being arbitrarily combined to create a bicluster. However, a special interpretation is given for tri-factorizations approximating the data matrix as known from checkerboard clustering (cf. \@Eq.~\eqref{eq:checkerOrig}) when biclusters represent a one-to-one relationship of row- and column-clusters. This task is also known under the name of constant biclustering~\citep{madeira2004biclustering}. The corresponding matrix factorization formulations follow from the equivalences of checkerboard objectives in Theorem~\ref{thm:checkerboard} and the observation that a one-to-one correspondence between row- and column-clusters implies that the middle scaling matrix $C$ is diagonal.
\begin{corollary}\label{thm:blockdiagonal}
Let the set of possible biclusters be given as
\begin{align*}
    \mathcal{B} = \{(\mathcal{I}_1,\mathcal{J}_1),\ldots, (\mathcal{I}_r,\mathcal{J}_r)\mid \dot\bigcup_{s}\mathcal{I}_s=\{1,\ldots,n\},\  \dot\bigcup_{s}\mathcal{J}_s=\{1,\ldots,m\}\}.
\end{align*}
and denote with the matrix $W$ the symmetric $(m+ n)\times (m+n)$ matrix defined as
\begin{align*}
   W=
\begin{pmatrix}
0 & D\\
D^\top & 0
\end{pmatrix}. 
\end{align*}
The following optimization problems are equivalent to the objective of Eq.~\eqref{eq:checkerAvg}, where $\mathcal{B}$ is defined as above:
\begin{align}
    \label{eq:blockDiagC}
    \min_{X,C,Y}&\ \|D-YCX^\top\|^2 &
    \text{s.t. } X\in \mathbbm{1}^{n\times r},\ Y\in\mathbbm{1}^{m\times r},\ C=\diag(\mathbf{c}),\ \mathbf{c}\in\mathbb{R}^{r}\\ 
    \label{eq:blockDiagAvg}
\min_{X,Y}&\ \|D-YCX^\top\|^2&
\text{s.t. } X\in \mathbbm{1}^{n\times r},\ Y\in\mathbbm{1}^{m\times r}, C_{ss}= \frac{Y_{\cdot s}^\top D X_{\cdot s}}{|Y_{\cdot s}||X_{\cdot s}|}, C_{st}=0, s\neq t\\
\label{eq:blockDiagtr}
\max_{X,Y}&\ \tr(D^\top YCX^\top) &
\text{s.t. } X\in \mathbbm{1}^{n\times r},\ Y\in\mathbbm{1}^{m\times r},\ C=\diag(\mathbf{c}),\ \mathbf{c}\in\mathbb{R}^{r}\\
\label{eq:blockDiagW}
\max_{Z}&\ \tr(Z^\top WZ) &
\text{s.t. }  Z=YC^{1/2},\ Y\in \mathbbm{1}^{(n+m)\times r},\ C=\diag(\mathbf{c}),\ \mathbf{c}\in\mathbb{R}^{r}
\end{align}
\end{corollary}
The tri-factorization $YCX^\top$ from Eq.~\@\eqref{eq:blockDiagC} boils down to the sum of $r$ outer products $Y_{\cdot s}X_{\cdot s}$, which are scaled by the diagonal entry $C_{ss}$. We call the corresponding tasks from Corollary~\ref{thm:blockdiagonal} block-diagonal clustering for the reason that a suitable reordering of rows and columns displays a block-diagonal factorization, as shown on the right in Figure~\ref{fig:blockclusters}.

Based on the equivalence of Eq.~\eqref{eq:blockDiagC} and \eqref{eq:blockDiagAvg} which goes back to~\cite{mirkin1995additive}, \cite{han2017bilateral} propose an alternating minimization for block-diagonal clustering.
Other optimization schemes emerge from the interpretation of 
block-diagonal models as bipartite graph cuts. In this view, the data matrix indicates a bipartite graph $\mathcal{G}=(\mathcal{V},\mathcal{E})$ having $|\mathcal{V}|=n+m$ nodes corresponding to the union of features and samples and edge weights given via the weighted adjacency matrix $W$.
%The bipartite graph view enables the application of graph cut methods, notably spectral clustering. 
If $W$ is defined as in Corollary~\ref{thm:blockdiagonal}, then the eigenvalues and eigenvectors of $W$ correspond to the singular values and vectors of $D$. To see that, assume that $(y,x)\in\mathbb{R}^{n+m}$ is an eigenvector of $W$ with eigenvalue $\lambda$; it holds that
\begin{align}
\begin{pmatrix}
0 & D\\
D^\top & 0
\end{pmatrix}
\begin{pmatrix}
y\\x
\end{pmatrix}
= \lambda \begin{pmatrix}
y\\ x
\end{pmatrix}\ 
&\Leftrightarrow Dx=\lambda y \ \wedge \ Dy=\lambda x\label{eq:singEigK}\\
&\Rightarrow DD^\top y = \lambda^2 y \wedge DD^\top x=\lambda^2 x. \nonumber
\end{align}
The equation above yields in combination with Eq.~\eqref{?} that eigenvalues of $W$ are also singular values of $D$. On the other hand, if $\lambda$ is a singular value of $D$ then the term on the right of Eq.~\eqref{eq:singEigK} holds for some vectors $x$ and $y$. Thus, $\lambda$ is also an eigenvalue of $W$ by the equivalence in Eq.~\eqref{eq:singEigK}. As a result, the eigenvalues of $W$ equate the singular values of $D$ and a similar argumentation yields that eigenvectors of $W$ correspond to left and right singular vectors of $D$. This relationship still holds if the weight matrix is normalized. Therefore, the eigenvectors of the $(n+m)\times (n+m)$ dimensional matrix $W$ are efficiently computed by the singular vectors of the $m\times n$ dimensional matrix $D$. 

The computation of $W$'s eigenvectors is relevant for a spectral relaxation, as originally proposed by~\cite{zha2001bipartite,dhillon2001co}. Newer methods minimizing bipartite graph cuts discuss how to learn the graph representation of $W$, respectively $D$, together with the optimal clustering~\citep{nie2017learning}. This trend has also been observed with respect to spectral clustering (cf. \@Section~\ref{sec:ZS:graphCut}).
The spectral relaxation emerges from the fact that the trace maximization in Eq.~\eqref{eq:blockDiagW} yields the eigenvectors to the largest eigenvalues of $W$ if the matrix $Z$ is only required to be real-valued and orthonormal. We discuss this relationship more in Section~\ref{sec:ZS:SpectralRelaxation}. Here, the eigenvectors are given by the singular vectors of $D$, thus we could also speak of a singular value relaxation instead. The relationship of block-diagonal clustering and SVD becomes also apparent when we regard the tri-factorization of Eq.~\eqref{eq:blockDiagC}. Relaxing the constraints of $X$ and $Y$ to orthogonality constraints for real-valued matrices $X$ and $Y$ results in a truncated SVD. In this case, the largest absolute singular values are denoted on the diagonal of matrix $C$.
%============================
% Binary Matrix Factorization
%============================
\subsection{Binary Matrix Factorization}\label{sec:ZS:BinaryMF}
A somewhat special case of biclustering arises if the data matrix is binary. This affects applications of collaborative filtering (a movie either is watched or not), text analysis (word occurrences are binary) or genome data analysis (considering, e.g. mutations). In this situation, a decomposition into binary matrices befits the interpretability of the result. We state the optimization problem of binary matrix factorization in its general form as
\begin{align}\label{eq:BinMF}
\min_{X,Y}\ &\|D-YX^\top\|^2 & \text{s.t. } X\in\{0,1\}^{n\times r},Y\in\{0,1\}^{m\times r}. \tag{BiMF}
\end{align}
The requirement that both factor matrices have orthogonal columns as known from block-diagonal clustering is too strict for most of the relevant applications. A movie might be watched by multiple user groups and correspondingly, groups of words and documents or mutations and patients do not follow a one-to-one relationship. \cite{li2005general} show that alternating minimization is possible nevertheless, if one of the matrices indicates a partition. We summarize the objectives equivalent to~\eqref{eq:BinMF} for the partially orthogonal case in the following theorem.
\begin{restatable}{theorem}{BMFpartition}\label{thm:BMFpartition}
Define $\Theta(A)$ as the set
\[\Theta(A) = \{B| B_{ji}=\theta(A_{ji}) \text{ for } A_{ji}\neq 1/2, B_{ji}\in\{0,1\} \text{ for } A_{ji}=1/2 \}\]
The following optimization problems are equivalent:
\begin{align}
    \label{eq:BinMF_YPartition}
    \min_{X,Y}& \|D-YX^\top\|^2 & \text{s.t. } X\in\{0,1\}^{n\times r},Y\in\mathbbm{1}^{m\times r}\\
    \label{eq:BinMF_YPartitionThetaX}
    \min_{Y}& \|D-YX^\top\|^2 & \text{s.t. } X\in\Theta(D^\top Y(Y^\top Y)^{-1}),Y\in\mathbbm{1}^{m\times r}\\
    \label{eq:BinMF_YPartititionTr}
    \max_{X,Y}& \tr(Y^\top (2D-\mathbf{1})X) & \text{s.t. } X\in\{0,1\}^{n\times r},Y\in\mathbbm{1}^{m\times r}
\end{align}
\end{restatable}
\cite{koyuturk2003proximus} propose the subsequent optimization of clusters to approximate solutions of the partially orthogonal binary matrix factorization in Eq.~\eqref{eq:BinMF_YPartitionThetaX}. The proposed algorithm \textsc{Proximus} optimizes rank-one factorizations of the general objective~\eqref{eq:BinMF} via alternating minimization. We will discuss this greedy approach more in detail in Section~\ref{sec:ZS:GreedyApproach}.
\cite{shen2009mining} remark that the results of \textsc{Proximus} are highly sensitive to the initialization. They propose a tailored initialization, returning the solution of Eq.~\eqref{eq:BinMF_YPartititionTr} when binary matrices are relaxed to have entries between zero and one.

\cite{zhang2007binary,zhang2010binary} aim at solving the general problem~\eqref{eq:BinMF}. They discuss a relaxation of binary to nonnegative matrices and derive an optimization scheme to determine suitable thresholds to discretize \ref{eq:NMF} solutions. In addition, they propose a multiplicative update algorithm for the optimization of NMF with an integrated penalization term for non-binary values (cf.\@ Section~\ref{sec:ZS:Penalty}). A follow-up paper discusses the application of these multiplicative updates for symmetric binary matrix factorizations~\citep{zhang2013overlapping}. 
%===============================
% Boolean Matrix Factorization
%===============================
\subsection{Boolean Matrix Factorization}\label{sec:ZS:BooleanMF}
The attempt to allow for more overlap between clusters of binary data is  is naturally incorporated if the factorization is computed in Boolean algebra. 
\begin{definition}[Boolean algebra]\label{def:BoolAlgebra}
Let $\mathcal{A}$ be a set, let $\oplus$ and $\odot$ be two binary relations and $\thickbar{\phantom{a}}$ an unary relation on the set $A$. The structure $(A,\oplus,\odot,\thickbar{\cdot})$ is called a Boolean algebra if the following requirements hold
\begin{enumerate}
    \item $(A,\oplus,\odot)$ is a commutative semiring:
    \begin{enumerate}
        \item $(A,\oplus)$ is a commutative monoid with identity element $0$
        \item $(A,\odot)$ is a commutative monoid with identity element $1$
        \item multiplication distributes over addition:
        \begin{equation*}
            a\odot(b\oplus c) = (a\odot b)\oplus (a\odot c) \forall a,b,c\in A
        \end{equation*}
        \item multiplication with $0$ annihilates: $a\odot 0=0$ for $a\in A$
    \end{enumerate}
    \item Addition distributes over multiplication:
    \begin{equation*}
        a\oplus(b\odot c) = (a\oplus b)\odot (a\oplus c) \forall a,b,c\in A
    \end{equation*}
    \item complementary elements: $a\odot \thickbar{a}=0$ and $a\oplus \thickbar{a}=1$ for $a\in A$.
\end{enumerate}
\end{definition}
We are in this section interested in the two-element algebra $(\{0,1\},\oplus,\odot,\thickbar{\phantom{a}})$, where we define the Boolean complement as $\thickbar{a}=1-a$. Note, that the only deviation of the arithmetic in two-element Boolean algebra to traditional operations is that $1\oplus 1=1$. From the definition of the Boolean algebra follows that the structure $(\{0,1\},\oplus,\odot)$ is a semiring, notably the element $1$ has no additive inverse. On this structure we can define vector operations such as addition, scalar multiplication and an inner product similarly to the Euclidean vector space. Then the linear mappings from the $n$-dimensional to the $m$-dimensional Boolean space are given by Boolean product of  a $m\times n$-dimensional binary matrix with the $n$-dimensional binary vector~\citep{gudder2009boolean}. Given $Y\in\{0,1\}^{m\times r}$ and $X\in\{0,1\}^{n\times r}$, the Boolean matrix product is either defined element-wise as the Boolean inner product of row and column vectors or as the sum of outer product matrices
\begin{align*}
    (Y\odot X^\top)_{ji}=Y_{j1}X_{i1}\oplus \ldots \oplus Y_{jr}X_{ir},&& Y\odot X^\top = Y_{\cdot 1}X_{\cdot 1}^\top\oplus \ldots \oplus Y_{\cdot r}X_{\cdot r}^\top.
\end{align*}
Other definitions of algebraic structures in the binary space are also thinkable as discussed in \cite{miettinen2015generalized}, yet these extensions are out of the scope of this paper.
The space defined via the Boolean operations has similar properties like the nonnegative space, which also lacks additive inverses. Like the nonnegative space, the Boolean space is closed under its matrix product which is contrasted by the binary matrix product, returning possibly nonbinary matrices. We now define the Boolean matrix factorization task as
\begin{align} \label{eq:BoolMF}
\min_{X,Y}\ &\|D-Y\odot X^\top\|^2 & \text{s.t. } X\in\{0,1\}^{n\times r},Y\in\{0,1\}^{m\times r}. \tag{BMF}
\end{align}
We express the Boolean product in elementary algebra via the Heaviside step function in order to avoid confusion of Boolean and numerical vector operations. In this manner we denote equivalent formulations of the Boolean matrix factorization problem.
\begin{restatable}{theorem}{BMFtheta}\label{thm:BMFtheta}
The following optimization problems are equivalent to problem~\eqref{eq:BoolMF}:
\begin{align}
    \label{eq:BoolMF_theta}
    \min_{X,Y}& \|D-\theta(YX^\top)\|^2 & \text{s.t. } X\in\{0,1\}^{n\times r},Y\in\{0,1\}^{m\times r}\\
    \label{eq:BoolMF_abs}
    \min_{X,Y}& |D-\theta(YX^\top)| & \text{s.t. } X\in\{0,1\}^{n\times r},Y\in\{0,1\}^{m\times r}\\
    \label{eq:BoolMF_thetaTr}
    \max_{X,Y}& \tr( (2D-\mathbf{1})\theta(XY^\top)) & \text{s.t. } X\in\{0,1\}^{n\times r},Y\in\{0,1\}^{m\times r}
\end{align}
\end{restatable}
%Note that the trace maximization problem corresponds to the binary matrix factorization problem from Eq.~\eqref{eq:BinMF_YPartititionTr}. 
The most popular method to approximate problem~\eqref{eq:BoolMF} is the greedy approach \textsc{Asso}~\citep{miettinen2008discrete}. Outer products are subsequently computed, determining a feature cluster $X_{\cdot s}$ first and optimizing its cluster assignment $Y_{\cdot s}$ afterwards, such that the approximization error is minimized. 

Boolean matrix factorization has a noteworthy relationship to frequent pattern mining~\citep{aggarwal2014frequent} and discussing \ref{eq:BoolMF} using the terminology of pattern mining often comes more natural. To this end, we shortly introduce the denotation of pattern mining. In this respect, the binary data matrix represents a transactional database of $m$ transactions and $n$ items. Every transaction corresponds to a row of the data matrix, indicating the items which are contained in the transaction. A set of items is called a pattern and we say a transaction supports a pattern if the pattern is a subset of the transaction. The support of a pattern is then the number of supporting transactions in the database. Regarding the factorization of problem~\eqref{eq:BoolMF}, every outer product indicates a pattern $X_{\cdot s}$ and its assigned transactions $Y_{\cdot s}$. 
A special case arises if the factor matrix $Y$ in problem~\eqref{eq:BoolMF} is restricted to the supporting sets of $X$.  This introduces the constraint $\langle D,\theta(YX^\top)\rangle=|\theta(YX^\top)|$ which enforces that all ones in the Boolean product matrix are covered by the data matrix. 
\begin{restatable}{corollary}{DominatedBMF}\label{thm:dominatedBMF}
The following optimization problems are equivalent:
%\begin{align*}
%    \langle D, \theta(YX^\top)\rangle =\tr(D\theta(XY^\top))=|\theta(YX^\top)| = \tr(\theta(YX^\top))
%\end{align*}
\begin{align}
    \label{eq:BoolMF_pattern}
    \min_{X,Y}&\ \|D-\theta(YX^\top)\|^2 & \text{s.t. } \langle D,\theta(YX^\top)\rangle = |\theta(YX^\top)|,X\in\{0,1\}^{n\times r}, Y\in\{0,1\}^{m\times r}\\
    \label{eq:BoolMF_patternTr}
    \max_{X,Y}&\ |\theta(YX^\top)|& \text{s.t. } \langle D,\theta(YX^\top)\rangle = |\theta(YX^\top)|,X\in\{0,1\}^{n\times r}, Y\in\{0,1\}^{m\times r}
\end{align}
\end{restatable}
\cite{geerts2004tiling} introduce the task of tiling to maximize the number of ones in the Boolean product according to Eq.~\eqref{eq:BoolMF_patternTr}. The term \emph{tile} for the outer product $Y_{\cdot s}X_{\cdot s}^\top$ reflects its visualization as a single block matrix for suitably arranged columns and rows. The task to determine the constrained Boolean matrix factorization in Eq.~\eqref{eq:BoolMF_pattern} is also known as dominated Boolean matrix factorization~\citep{miettinen2010sparse}. \cite{belohlavek2010discovery,belohlavek2015below} propose a greedy approach for dominant matrix factorizations based on an approximation algorithm of the set cover problem. 
\cite{kontonasios2010information} and \cite{xiang2011summarizing} argue that lifting the restriction to the support sets enables more succinct descriptions and enhances robustness to noise; every flip of a single bit in the interior of a tile breaks it into two. The proposed algorithms handle the extension from the supporting set only in a post-processing step and provide no mechanisms to directly approximate a solution to Eq.~\eqref{eq:BoolMF_pattern}.

